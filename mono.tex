%% Produzido a partir de:
%% abtex2-modelo-artigo.tex, v-1.9.5 laurocesar
%% Copyright 2012-2015 by abnTeX2 group at http://www.abntex.net.br/
% ------------------------------------------------------------------------
% ------------------------------------------------------------------------
% abnTeX2: Modelo de Artigo Acadêmico em conformidade com
% ABNT NBR 6022:2003: Informação e documentação - Artigo em publicação
% periódica científica impressa - Apresentação
% ------------------------------------------------------------------------
% ------------------------------------------------------------------------

\documentclass[
% -- opções da classe memoir --
article,			% indica que é um artigo acadêmico
11pt,				% tamanho da fonte
oneside,			% para impressão apenas no verso. Oposto a twoside
a4paper,			% tamanho do papel.
% -- opções da classe abntex2 --
%chapter=TITLE,		% títulos de capítulos convertidos em letras maiúsculas
%section=TITLE,		% títulos de seções convertidos em letras maiúsculas
%subsection=TITLE,	% títulos de subseções convertidos em letras maiúsculas
%subsubsection=TITLE % títulos de subsubseções convertidos em letras maiúsculas
% -- opções do pacote babel --
english,			% idioma adicional para hifenização
brazil,				% o último idioma é o principal do documento
sumario=tradicional
]{abntex2}


% ---
% PACOTES
% ---

% ---
% Pacotes fundamentais
% ---
\usepackage{lmodern}			% Usa a fonte Latin Modern
\usepackage[T1]{fontenc}		% Selecao de codigos de fonte.
\usepackage[utf8]{inputenc}		% Codificacao do documento (conversão automática dos acentos)
\usepackage{indentfirst}		% Indenta o primeiro parágrafo de cada seção.
\usepackage{nomencl} 			% Lista de simbolos
\usepackage{color}				% Controle das cores
\usepackage{graphicx}			% Inclusão de gráficos
\usepackage{microtype} 			% para melhorias de justificação
% ---

% ---
% Pacotes de diagrama
% ---
\usepackage{smartdiagram}
% ---
% Pacotes adicionais, usados apenas no âmbito do Modelo Canônico do abnteX2
% ---
\usepackage{lipsum}				% para geração de dummy text
% ---

% ---
% Pacotes de citações
% ---
\usepackage[brazilian,hyperpageref]{backref}	 % Paginas com as citações na bibl
\usepackage[alf]{abntex2cite}	% Citações padrão ABNT
% ---

% ---
% Configurações do pacote backref
% Usado sem a opção hyperpageref de backref
\renewcommand{\backrefpagesname}{Citado na(s) página(s):~}
% Texto padrão antes do número das páginas
\renewcommand{\backref}{}
% Define os textos da citação
\renewcommand*{\backrefalt}[4]{
\ifcase #1 %
Nenhuma citação no texto.%
\or
Citado na página #2.%
\else
Citado #1 vezes nas páginas #2.%
\fi}%
% ---

% ---
% Informações de dados para CAPA e FOLHA DE ROSTO
% ---
\titulo{Tópicos de pesquisa sobre narrativas transmídia em Computação}
\autor{Yuri Soares de Oliveira\thanks{soaresyuri@gmail.com} \and Cidcley Teixeira de Souza\thanks{cidcley@ifce.edu.br}}
\local{Fortaleza, Brasil}
\data{Dezembro 2015}
% ---

% ---
% Configurações de aparência do PDF final

% alterando o aspecto da cor azul
\definecolor{blue}{RGB}{41,5,195}

% informações do PDF
\makeatletter
\hypersetup{
%pagebackref=true,
pdftitle={\@title},
pdfauthor={\@author},
pdfsubject={Tópicos de Pesquisa Sobre Narrativas Transmídia em Computação},
pdfcreator={LaTeX with abnTeX2},
pdfkeywords={abnt}{latex}{abntex}{abntex2}{artigo científico},
colorlinks=true,       		% false: boxed links; true: colored links
linkcolor=blue,          	% color of internal links
citecolor=blue,        		% color of links to bibliography
filecolor=magenta,      		% color of file links
urlcolor=blue,
bookmarksdepth=4
}
\makeatother
% ---

% ---
% compila o indice
% ---
\makeindex
% ---

% ---
% Altera as margens padrões
% ---
\setlrmarginsandblock{3cm}{3cm}{*}
\setulmarginsandblock{3cm}{3cm}{*}
\checkandfixthelayout
% ---

% ---
% Espaçamentos entre linhas e parágrafos
% ---

% O tamanho do parágrafo é dado por:
\setlength{\parindent}{1.3cm}

% Controle do espaçamento entre um parágrafo e outro:
\setlength{\parskip}{0.2cm}  % tente também \onelineskip

% Espaçamento simples
\SingleSpacing

% ----
% Início do documento
% ----
\begin{document}

  % Seleciona o idioma do documento (conforme pacotes do babel)
  %\selectlanguage{english}
  \selectlanguage{brazil}

  % Retira espaço extra obsoleto entre as frases.
  \frenchspacing

  % ----------------------------------------------------------
  % ELEMENTOS PRÉ-TEXTUAIS
  % ----------------------------------------------------------

  %---
  %
  % Se desejar escrever o artigo em duas colunas, descomente a linha abaixo
  % e a linha com o texto ``FIM DE ARTIGO EM DUAS COLUNAS''.
  % \twocolumn[    		% INICIO DE ARTIGO EM DUAS COLUNAS
  %
  %---
  % página de titulo
  \maketitle

  % resumo em português
  \begin{resumoumacoluna}
    Esse estudo revisa a literatura de publicações científicas sobre narrativas transmídia em Computação a fim de (a) identificar os tópicos mais pesquisados e (b) as áreas de aplicação mais estudadas, a fim de contribuir para a visualização e melhor entendimento do atual estado da arte das pesquisas em narrativas transmídia. Os artigos revisados sugerem que \textsf{RESUMIR A DISCUSSÃO E CONCLUSÃO AQUI} (\ldots) Alguma coisa a mais aqui no resumo.

    \vspace{\onelineskip}

    \noindent
    \textbf{Palavras-chave}: narrativa transmídia. transmídia. revisão sistemática.
  \end{resumoumacoluna}

  % ]  				% FIM DE ARTIGO EM DUAS COLUNAS
  % ---

  % ----------------------------------------------------------
  % ELEMENTOS TEXTUAIS
  % ----------------------------------------------------------
  \textual

  % ----------------------------------------------------------
  % Introdução
  % ----------------------------------------------------------
  % O que é narrativa transmídia / Diferença entre transmídia, crossmídia, multimídia
  % narrativa transmídia e a pesquisa em computação
  % O uso de ferramentas/frameworks/processos/conceitos para que a narrativa (storytelling) possa ser extendida (ampliada/explorada) a múltiplos suportes ou mídias, especialmente as digitais
  \section{Introdução}

  A \textit{Internet} possibilita, além de conectar várias pessoas simultaneamente, agregar meios de comunicação, integrando-os e ampliando-lhes seu potencial. A convergência produzida através dessa agregação permitiu o surgimento de uma ``Renascença digital'', e com ela novas maneiras de se contar histórias através de diversos meios \cite{jenkins_2001}. A emergência desse modelo inovador de narração teve início com a televisão contemporânea, especialmente no âmbito da produção seriada televisiva norte-americana. Esse processo de reconfiguração narrativa se deu de forma simultânea ao desenvolvimento das tecnologias de reprodução e armazenamento de dados, notadamente as plataformas de TV digital, de segunda tela e os jogos – Smart TV, smartphones e tablets, consoles de \textit{videogame} e outros dispositivos.

  A narrativa transmídia é o processo de dispersar sistematicamente elementos de um enredo em múltiplas plataformas, permitindo que cada um contribua para o todo. Cada meio realiza sua função: as histórias em quadrinhos fornecem a história geral, os jogos permitem explorar o mundo criado e as séries de televisão oferecem desdobramentos narrativos diferentes, por exemplo \cite{jenkins_fastcompany_2011}.

  Com base na conceituação antecedente, \citeonline{scolari_2008} define narrativa transmídia como uma estrutura que se expande tanto em termos de linguagens (verbais, icônicas, textuais etc) quanto de mídias (televisão, rádio, celular, internet, jogos, quadrinhos e outros). As histórias se complementam em cada suporte e devem fazer sentido isoladamente, conforme propõe Jenkins. O autor ressalta que o termo ainda gera muita confusão. \citeonline{hanson_2004}, por exemplo, refere-se ao termo ``screen bleed'' para nomear universos ficcionais que ultrapassam os limites de sua mídia, indo além dos limites da tela. A pesquisadora \citeonline{dena_2004} cunhou o termo ``transficção''  para  designar  uma  mesma  história  distribuída  por  diferentes  mídias. O presente trabalho utiliza o termo ``narrativa transmídia'' como tradução do termo ``transmedia storytelling'' cunhado por \citeonline{jenkins_2003} e por sua definição ser suficientemente abrangente para abrigar os termos anteriormente citados.

  A discussão central a ser desenvolvida nesse estudo é possibilitar melhor visualização e entendimento do atual estado da arte das pesquisas em narrativas transmídia.  O objetivo desse estudo é conduzir uma revisão sistemática acerca de narrativas transmídia em Computação a fim de:

  (a) identificar os tópicos mais pesquisados em narrativas transmídias,

  (b) identificar as áreas de aplicação mais estudadas,

  (c) classificar as pesquisas em narrativas transmídias com base nos tópicos e áreas de aplicação.

  \section{Métodos}

  \textsf{(definir revisão sistemática, suas etapas e o que foi realizado em cada uma)}

  Uma revisão sistemática é um método que possibilita a avaliação e interpretação de toda a pesquisa relevante acessível para uma ou mais questões de pesquisa ou evento de interesse. Para essa revisão, seguiu-se um processo definido para a condução de revisões sistemáticas proposto por \citeonline{kit_cha_2007}:

  \begin{description}
    \item[Etapa 1: Planejamento da revisão] \hfill \\
    Atividade 1.1: Identificação da necessidade de uma revisão \\
    Atividade 1.2: Desenvolvimento de um protocolo de revisão
    \item[Etapa 2: Condução da revisão] \hfill \\
    Atividade 2.1: Identificação da busca \\
    Atividade 2.2: Seleção de estudos primários \\
    Atividade 2.3: Estudo de qualidade \\
    Atividade 2.4: Extração de dados \\
    Atividade 2.5: Sintetização de dados
    \item[Etapa 3: Relatando a revisão] \hfill \\
    Atividade 3.1: Comunicando os resultados
  \end{description}

  \subsection{Planejamento e Condução da revisão}

  (\ldots)

  \emph{Questão 1: Quais tópicos são mais pesquisados sobre narrativas transmídias?}

  \emph{Questão 2: Quais os objetivos mais apresentados?}

  \emph{Questão 3: Que perspectiva de pesquisas futuras se pode inferir sobre narrativas transmídia?}

  (\ldots)

  A \textit{string} de busca utilizada foi: ((narrativ* OR storytelling OR ``digital storytelling'') AND (``second screen'' OR multiscreen OR interactive OR transmedia OR crossmedia OR ``cross media'')). A \autoref{tab-queries2} apresenta o protocolo utilizado em cada base de dados.

  \begin{table}[htb]
    \ABNTEXfontereduzida
    \caption[\textit{Strings} de busca]{\textit{Strings} de busca.}
    \label{tab-queries2}
    \begin{tabular}{p{2.0cm}|p{9.3cm}|p{2.8cm}}
      %\hline
      \textbf{Fonte} & \textbf{\textit{String} de busca} & \textbf{Nota} \\
      \hline
      ACM & Title:((narrativ* OR storytelling OR ``digital storytelling'') AND (``second screen'' OR multiscreen OR interactive OR transmedia OR crossmedia OR ``cross media'')) OR Abstract:((narrativ* OR storytelling OR ``digital storytelling'') AND (``second screen'' OR multiscreen OR interactive OR transmedia OR crossmedia OR ``cross media'')) & Busca em ``Advanced Search'', filtro de data (2009 a 2015) adicionado manualmente \\
      \hline
      IEEE & (narrativ* OR storytelling OR ``digital storytelling'') AND (``second screen'' OR multiscreen OR interactive OR transmedia OR crossmedia OR ``cross media'') & Busca em ``Command Search'', filtro de data adicionado manualmente \\
      \hline
      ScienceDirect & pub-date > 2008 and tak((narrativ* OR storytelling OR ``digital storytelling'') AND (``second screen'' OR multiscreen OR interactive OR transmedia OR crossmedia OR ``cross media'')) [All Sources(Computer Science,Engineering)]) & Busca em ``Advanced search'', filtros ``pub-date'' e ``All Sources'' adicionados manualmente \\
      \hline
      Scopus & TITLE-ABS ( ( narrativ*  OR  storytelling  OR  ``digital storytelling'' )  AND  ( ``second screen''  OR  multiscreen  OR  interactive  OR  transmedia  OR  crossmedia  OR  ``cross media'' ) )  AND  ( LIMIT-TO ( SUBJAREA ,  ``COMP'' )  OR  LIMIT-TO ( SUBJAREA ,  ``ENGI'' ) )  AND  ( LIMIT-TO ( PUBYEAR ,  2016 )  OR  LIMIT-TO ( PUBYEAR ,  2015 )  OR  LIMIT-TO ( PUBYEAR ,  2014 )  OR  LIMIT-TO ( PUBYEAR ,  2013 )  OR  LIMIT-TO ( PUBYEAR ,  2012 )  OR  LIMIT-TO ( PUBYEAR ,  2011 )  OR  LIMIT-TO ( PUBYEAR ,  2010 )  OR  LIMIT-TO ( PUBYEAR ,  2009 ) ) & Busca em ``Advanced search'', filtros ``LIMIT-TO'' adicionados manualmente \\
      % \hline
    \end{tabular}
    \legend{Fonte: Produzido pelos autores.}
  \end{table}

  Critérios de inclusão (I) e de exclusão (E):

  (E) Não está relacionado ao tema narrativas transmídia

  (E) O resumo do artigo não esclarece o aspecto transmídia desenvolvido no trabalho

  (I) Trata de conceitos

  (I) Trata de processos

  (I) Trata de ferramentas

  (I) Trata de sincronismo de dados

  (I) Estudo de caso

  A \autoref{tab-article-selection} resume a etapa de seleção.

  \begin{table}[htb]
    \ABNTEXfontereduzida
    \caption[Etapa de seleção]{Etapa de seleção.}
    \label{tab-article-selection}
    \begin{tabular}{p{3.0cm}|p{2.0cm}|p{1.0cm}|p{1.0cm}|p{1.0cm}|p{1.0cm}|p{1.0cm}|p{2.0cm}}
      %\hline
      \textbf{Fonte} & \textbf{Resultantes da busca} & \textbf{CI1} & \textbf{CI2} & \textbf{CI3} & \textbf{CI4} & \textbf{CI5} & \textbf{Selecionados}  \\
      \hline
      ACM Digital Library & 240 & 9 & 4 & 6 & 1 & 12 & 32 \\
      \hline
      IEEE Xplore & 187 & 0 & 1 & 2 & 0 & 7 & 10 \\
      \hline
      ScienceDirect & 35 & 0 & 1 & 1 & 0 & 3 & 5 \\
      \hline
      Scopus & 970 & 12 & 4 & 4 & 0 & 30 & 50 \\
      \hline
      \textbf{Total} & \textbf{1432} & \textbf{21} & \textbf{10} & \textbf{13} & \textbf{1} & \textbf{52} & \textbf{97} \\
      % \hline
    \end{tabular}
    \legend{Fonte: Produzido pelos autores.}
  \end{table}

  Comentários adicionais: na seleção: na extração: 97 aceitos, 19 com alta prioridade de leitura, 44 com prioridade

  \section{Resultados}
  equanto a \autoref{diag-trata-de-para} ..

  \begin{figure}[htb]
    \caption{\label{diag-trata-de-para}Categorias de classificação}
    \begin{center}
      \smartdiagramset{font=\scriptsize, module minimum width=2.5cm, uniform arrow color=true, module x sep=4.0cm, back arrow disabled, uniform color list=white for 3 items}
      \smartdiagram[flow diagram:horizontal]{Conceitos \\ Processos \\ Ferramentas \\ Sincronismo de dados \\ Estudo de caso \\, Planejar \\ Implementar \\ Analisar \\ Adaptar, Narrativas transmídia} \\
    \end{center}
    \legend{Fonte: Produzido pelos autores}
  \end{figure}

  A \autoref{tab-trata-de} apresenta a classificação.

  \begin{table}[htb]
    \ABNTEXfontereduzida
    \caption[Categorias de classificação]{Categorias de classificação.}
    \label{tab-trata-de}
    \begin{center}
      \begin{tabular}{p{3.0cm}|p{2.0cm}}
        %\hline
        \textbf{Trata de} & \textbf{Artigos} \\
        \hline
        Conceitos & 21 \\
        \hline
        Processos & 10 \\
        \hline
        Ferramentas & 13\\
        \hline
        Sincronismo & 1\\
        \hline
        Estudo de Caso & 52\\
        % \hline
        \hline
        \textbf{Total} & \textbf{97} \\
      \end{tabular}
    \end{center}
    \legend{Fonte: Produzido pelos autores.}
  \end{table}

  A \autoref{tab-para} apresenta os objetivos.

  \begin{table}[htb]
    \ABNTEXfontereduzida
    \caption[Objetivos]{Objetivos.}
    \label{tab-para}
    \begin{center}
      \begin{tabular}{p{3.0cm}|p{2.0cm}}
        %\hline
        \textbf{Objetivo} & \textbf{Artigos} \\
        \hline
        Planejar & 20 \\
        \hline
        Implementar & 52 \\
        \hline
        Analisar & 14 \\
        \hline
        Adaptar & 11 \\
        \hline
        \textbf{Total} & \textbf{97} \\
        % \hline
      \end{tabular}
    \end{center}
    \legend{Fonte: Produzido pelos autores.}
  \end{table}

  A \autoref{tab-conceitos} apresenta os conceitos.

  \begin{table}[htb]
    \ABNTEXfontereduzida
    \caption[Conceitos x Objetivos]{Conceitos x Objetivos.}
    \label{tab-conceitos}
    \begin{center}
      \begin{tabular}{p{3.0cm}|p{2.0cm}|p{.5cm}}
        %\hline
        \textbf{Artigo apresenta} & \textbf{Para} & \\
        \hline
        Conceitos & Planejar \newline Implementar \newline Analisar \newline Adaptar & 13 \newline 1 \newline 6 \newline 1 \\
        \hline
        \textbf{Total} & & \textbf{21} \\
        % \hline
      \end{tabular}
    \end{center}
    \legend{Fonte: Produzido pelos autores.}
  \end{table}

  A \autoref{tab-processos} apresenta os processos.

  \begin{table}[htb]
    \ABNTEXfontereduzida
    \caption[Processos x Objetivos]{Processos x Objetivos.}
    \label{tab-processos}
    \begin{center}
      \begin{tabular}{p{3.0cm}|p{2.0cm}|p{.5cm}}
        %\hline
        \textbf{Artigo apresenta} & \textbf{Para} & \\
        \hline
        Processos & Planejar \newline Implementar \newline Analisar \newline Adaptar & 2 \newline 5 \newline 2 \newline 1 \\
        \hline
        \textbf{Total} & & \textbf{10} \\
        % \hline
      \end{tabular}
    \end{center}
    \legend{Fonte: Produzido pelos autores.}
  \end{table}

  A \autoref{tab-ferramentas} apresenta as ferramentas.

  \begin{table}[htb]
    \ABNTEXfontereduzida
    \caption[Ferramentas x Objetivos]{Ferramentas x Objetivos.}
    \label{tab-ferramentas}
    \begin{center}
      \begin{tabular}{p{3.0cm}|p{2.0cm}|p{.5cm}}
        %\hline
        \textbf{Artigo apresenta} & \textbf{Para} & \\
        \hline
        Ferramentas & Planejar \newline Implementar \newline Analisar \newline Adaptar & 1 \newline 11 \newline 0 \newline 1 \\
        \hline
        \textbf{Total} & & \textbf{13} \\
        % \hline
      \end{tabular}
    \end{center}
    \legend{Fonte: Produzido pelos autores.}
  \end{table}

  A \autoref{tab-estudo} apresenta os estudos de caso.

  \begin{table}[htb]
    \ABNTEXfontereduzida
    \caption[Estudos de caso x Objetivos]{Estudos de caso x Objetivos.}
    \label{tab-estudo}
    \begin{center}
      \begin{tabular}{p{3.0cm}|p{2.0cm}|p{.5cm}}
        %\hline
        \textbf{Artigo apresenta} & \textbf{Para} & \\
        \hline
        Estudos de caso & Planejar \newline Implementar \newline Analisar \newline Adaptar & 4 \newline 35 \newline 6 \newline 7 \\
        \hline
        \textbf{Total} & & \textbf{52} \\
        % \hline
      \end{tabular}
    \end{center}
    \legend{Fonte: Produzido pelos autores.}
  \end{table}

  \section{Discussão}

  Trabalhar a questão de que foram encontrados muitos resultados duplicados, muitos estudos de caso, somente uma pesquisa abordando a questão da sincronização dos dados


  \section{Conclusão}

  % ---
  % Finaliza a parte no bookmark do PDF, para que se inicie o bookmark na raiz
  % ---
  \bookmarksetup{startatroot}%
  % ---

  % ----------------------------------------------------------
  % ELEMENTOS PÓS-TEXTUAIS
  % ----------------------------------------------------------
  \postextual

  % ---
  % Título e resumo em língua estrangeira
  % ---

  % \twocolumn[    		% INICIO DE ARTIGO EM DUAS COLUNAS

  % titulo em inglês
  \titulo{Research topics on transmedia storytelling}
  \emptythanks
  \maketitle

  % resumo em português
  \renewcommand{\resumoname}{Abstract}
  \begin{resumoumacoluna}
    \begin{otherlanguage*}{english}
      This paper discuss a systematic review on research topics in transmedia storytelling. The chosen sources were: ACM, IEEE, ScienceDirect, Scopus

      \vspace{\onelineskip}

      \noindent
      \textbf{Keywords}: latex. abntex.
    \end{otherlanguage*}
  \end{resumoumacoluna}

  % ]  				% FIM DE ARTIGO EM DUAS COLUNAS
  % ---

  % ----------------------------------------------------------
  % Referências bibliográficas
  % ----------------------------------------------------------
  \bibliography{mono}

\end{document}
