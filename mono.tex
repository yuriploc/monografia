%% Modificado de: abtex2-modelo-artigo.tex, v-1.9.5 laurocesar
%% Copyright 2012-2015 by abnTeX2 group at http://www.abntex.net.br/
% ------------------------------------------------------------------------
% ------------------------------------------------------------------------
% abnTeX2: Modelo de Artigo Acadêmico em conformidade com
% ABNT NBR 6022:2003: Informação e documentação - Artigo em publicação
% periódica científica impressa - Apresentação
% ------------------------------------------------------------------------
% ------------------------------------------------------------------------

\documentclass[
% -- opções da classe memoir --
article,			% indica que é um artigo acadêmico
11pt,				% tamanho da fonte
oneside,			% para impressão apenas no verso. Oposto a twoside
a4paper,			% tamanho do papel.
% -- opções da classe abntex2 --
%chapter=TITLE,		% títulos de capítulos convertidos em letras maiúsculas
%section=TITLE,		% títulos de seções convertidos em letras maiúsculas
%subsection=TITLE,	% títulos de subseções convertidos em letras maiúsculas
%subsubsection=TITLE % títulos de subsubseções convertidos em letras maiúsculas
% -- opções do pacote babel --
english,			% idioma adicional para hifenização
brazil,				% o último idioma é o principal do documento
sumario=tradicional
]{abntex2}

% ---
% PACOTES
% ---

% ---
% Pacotes fundamentais
% ---
\usepackage{lmodern}			% Usa a fonte Latin Modern
\usepackage[T1]{fontenc}		% Selecao de codigos de fonte.
\usepackage[utf8]{inputenc}		% Codificacao do documento (conversão automática dos acentos)
\usepackage{indentfirst}		% Indenta o primeiro parágrafo de cada seção.
\usepackage{nomencl} 			% Lista de simbolos
\usepackage{color}				% Controle das cores
\usepackage{graphicx}			% Inclusão de gráficos
\usepackage{microtype} 			% para melhorias de justificação
\usepackage{longtable}
% ---

% ---
% Pacotes de diagrama
% ---
\usepackage{smartdiagram}
% ---

% ---
% Pacotes de citações
% ---
\usepackage[brazilian,hyperpageref]{backref}	 % Paginas com as citações na bibl
\usepackage[alf]{abntex2cite}	% Citações padrão ABNT
% ---

% ---
% Configurações do pacote backref
% Usado sem a opção hyperpageref de backref
\renewcommand{\backrefpagesname}{Citado na(s) página(s):~}
% Texto padrão antes do número das páginas
\renewcommand{\backref}{}
% Define os textos da citação
\renewcommand*{\backrefalt}[4]{
\ifcase #1 %
Nenhuma citação no texto.%
\or
Citado na página #2.%
\else
Citado #1 vezes nas páginas #2.%
\fi}%
% ---

% ---
% Informações de dados para CAPA e FOLHA DE ROSTO
% ---
\titulo{Tópicos de pesquisa sobre narrativas transmídia em Computação: uma revisão sistemática}
\autor{Yuri Soares de Oliveira\thanks{soaresyuri@gmail.com} \and Cidcley Teixeira de Souza\thanks{cidcley@ifce.edu.br}}
\local{Fortaleza, Brasil}
\data{Dezembro 2015}
% ---

% ---
% Configurações de aparência do PDF final

% alterando o aspecto da cor azul
\definecolor{blue}{RGB}{41,5,195}

% informações do PDF
\makeatletter
\hypersetup{
%pagebackref=true,
pdftitle={\@title},
pdfauthor={\@author},
pdfsubject={Tópicos de Pesquisa Sobre Narrativas Transmídia em Computação},
pdfcreator={LaTeX with abnTeX2},
pdfkeywords={abnt}{latex}{abntex}{abntex2}{artigo científico},
colorlinks=true,       		% false: boxed links; true: colored links
linkcolor=blue,          	% color of internal links
citecolor=blue,        		% color of links to bibliography
filecolor=magenta,      		% color of file links
urlcolor=blue,
bookmarksdepth=4
}
\makeatother
% ---

% ---
% compila o indice
% ---
\makeindex
% ---

% ---
% Altera as margens padrões
% ---
\setlrmarginsandblock{3cm}{3cm}{*}
\setulmarginsandblock{3cm}{3cm}{*}
\checkandfixthelayout
% ---

% ---
% Espaçamentos entre linhas e parágrafos
% ---

% O tamanho do parágrafo é dado por:
\setlength{\parindent}{1.3cm}

% Controle do espaçamento entre um parágrafo e outro:
\setlength{\parskip}{0.2cm}  % tente também \onelineskip

% Espaçamento simples
\SingleSpacing

% ----
% Início do documento
% ----
\begin{document}

  \selectlanguage{brazil}

  % Retira espaço extra obsoleto entre as frases.
  \frenchspacing

  % ----------------------------------------------------------
  % ELEMENTOS PRÉ-TEXTUAIS
  % ----------------------------------------------------------

  %---
  % página de titulo
  \maketitle

  % resumo em português
  \begin{resumoumacoluna}
    Esse estudo revisa a literatura de publicações científicas sobre narrativas transmídia em Computação a fim de (a) identificar os tópicos e objetivos mais pesquisados em narrativas transmídias e (b) classificar as pesquisas em narrativas transmídias com base nos tópicos e objetivos encontrados a fim de contribuir para a visualização e melhor entendimento do atual estado da arte das pesquisas em narrativas transmídia. Os artigos revisados sugerem que \textsf{RESUMIR A DISCUSSÃO E CONCLUSÃO AQUI} (\ldots) Alguma coisa a mais aqui no resumo.

    \vspace{\onelineskip}

    \noindent
    \textbf{Palavras-chave}: narrativa transmídia. transmídia. revisão sistemática.
  \end{resumoumacoluna}

  % ----------------------------------------------------------
  % ELEMENTOS TEXTUAIS
  % ----------------------------------------------------------
  \textual

  % ----------------------------------------------------------
  % Introdução
  % ----------------------------------------------------------
  % O que é narrativa transmídia / Diferença entre transmídia, crossmídia, multimídia
  % narrativa transmídia e a pesquisa em computação
  % O uso de ferramentas/frameworks/processos/conceitos para que a narrativa (storytelling) possa ser extendida (ampliada/explorada) a múltiplos suportes ou mídias, especialmente as digitais
  \section{Introdução}

  A \textit{Internet} possibilita, além de conectar várias pessoas simultaneamente, agregar meios de comunicação, integrando-os e ampliando seu potencial. A convergência produzida por essa agregação permitiu o surgimento de uma ``Renascença digital'', e com ela novas maneiras de se contar histórias que se expandem por meios diversos \cite{jenkins_2001}. Esse processo de reconfiguração da narrativa se deu de forma simultânea ao desenvolvimento das tecnologias de reprodução e armazenamento de dados, notadamente as plataformas de TV digital, de segunda tela e os jogos, através de Smarts TV, \textit{smartphones}, \textit{tablets}, consoles de \textit{videogame} e outros dispositivos.

  Um dos conceitos mais presentes dessa ``Renascença digital'' é a narrativa transmídia, proposta como o processo de dispersar sistematicamente elementos de um enredo em múltiplas plataformas, permitindo que cada um contribua para o todo. Cada meio realiza sua função: as histórias em quadrinhos fornecem a história geral, os jogos permitem explorar o mundo criado e as séries de televisão oferecem desdobramentos narrativos diferentes, por exemplo \cite{jenkins_fastcompany_2011}.

  Com base no conceito anterior, \citeonline{scolari_2008} define narrativa transmídia como uma estrutura que se expande tanto em termos de linguagens (verbais, icônicas, textuais etc) quanto de mídias (televisão, rádio, celular, internet, jogos, quadrinhos e outros). As histórias se complementam em cada suporte e devem fazer sentido isoladamente, conforme propõe \citeonline{jenkins_2003}, apesar de que o termo ainda gera confusão. \citeonline{hanson_2004}, por exemplo, refere-se ao termo \textit{``screen bleed''} para nomear universos ficcionais que ultrapassam os limites de sua mídia, indo além dos limites da tela. \citeonline{dena_2004} cunhou o termo ``transficção''  para  designar  uma  mesma  história  distribuída  por  diferentes  mídias. O presente trabalho utiliza o termo ``narrativa transmídia'' como tradução do termo \textit{``transmedia storytelling''} desenvolvido por \citeonline{jenkins_2003} e por sua definição ser suficientemente abrangente para abrigar os termos anteriormente citados.

  Diversos trabalhos em Computação têm sido publicados a respeito de narrativas transmídias e suas aplicações: design, análise ou implementação de jogos de realidade aumentada, seja para entretenimento, saúde ou educação \cite{bonsignore_2012,evans_2014,johnston_2012,holler_2014}; aplicativos de segunda tela sincronizados com seriados de TV \cite{nandakumar_2014}; utilização de jogos pelo Exército norte-americano para treinamentos diversos \cite{raybourn_2014} e análise transmídia das telenovelas brasileiras \cite{murakami_2015} são alguns exemplos.

  Essa diversidade de aplicações pode produzir incertezas sobre como a Computação tem contribuído para o desenvolvimento desse conceito multidisciplinar. Considerando a falta de revisões e classificações sobre narrativas transmídias, a discussão central a ser desenvolvida nesse trabalho é possibilitar uma melhor visualização e entendimento do atual estado da arte das pesquisas sobre o tema.  O objetivo desse estudo é conduzir uma revisão sistemática acerca de narrativas transmídia em Computação a fim de:

  (a) identificar os tópicos e objetivos mais pesquisados em narrativas transmídias,

  (b) classificar as pesquisas em narrativas transmídias com base nos tópicos e objetivos encontrados.

  % ----------------------------------------------------------
  % Métodos
  % ----------------------------------------------------------
  % definir revisão sistemática
  % suas etapas
  % o que foi realizado em cada etapa
  % explicar as categorias de classificação
  \section{Métodos}
  \label{sec-metodos}

  Uma revisão sistemática é um método que possibilita a avaliação e interpretação de toda a pesquisa relevante acessível para uma ou mais questões de pesquisa ou evento de interesse. Para esse artigo, seguiu-se um processo definido para a condução de revisões sistemáticas proposto por \citeonline{kit_cha_2007}:

  \begin{description}
    \item[Etapa 1: Planejamento da revisão] \hfill \\
    Atividade 1.1: Identificação da necessidade de uma revisão \\
    Atividade 1.2: Desenvolvimento de um protocolo de revisão
    \item[Etapa 2: Condução da revisão] \hfill \\
    Atividade 2.1: Identificação da busca \\
    Atividade 2.2: Seleção de estudos primários \\
    Atividade 2.3: Estudo de qualidade \\
    Atividade 2.4: Extração de dados \\
    Atividade 2.5: Sintetização de dados
    \item[Etapa 3: Relatando a revisão] \hfill \\
    Atividade 3.1: Comunicando os resultados
  \end{description}

  \subsection{Planejamento e Condução da revisão}
  \label{subsec-planejamento}

  Como etapa prévia ao planejamento, conduziu-se uma pesquisa manual para identificar revisões sistemáticas sobre narrativas transmídias, e nenhum trabalho específico sobre o tema foi encontrado. Entretanto, trabalhos como o de \citeonline{murray_2012} e de \citeonline{amy_2014} resultaram dessa busca. O primeiro desenvolve o conceito de convergência dos meios digitais a fim de produzir narrativas transmídias que sejam imersivas e significativas, enquanto o segundo propõe um modelo de análise de projetos transmídia.

  No contexto desse artigo, utilizou-se a abordagem básica para a realização de uma revisão sistemática identificada em \citeonline{kit_cha_2007}, a fim de realizar os objetivos mencionados anteriormente, baseando-se nas seguintes questões de pesquisa:

  \emph{Questão 1: Quais tópicos são mais pesquisados sobre narrativas transmídias?}

  \emph{Questão 2: Quais os objetivos mais apresentados?}

  % TODO: DÚVIDA %\emph{Questão 3: Que perspectiva de pesquisas futuras se pode inferir sobre narrativas transmídia?}

  Para o propósito desse estudo, a revisão sistemática foi realizada entre novembro e dezembro de 2015, nas seguintes bases de dados internacionais: (a) ACM Digital Library, (b) IEEE Xplore, (c) ScienceDirect e (d) Scopus. As buscas foram restritas a trabalhos publicados em língua inglesa entre 2009 e 2015, e nas bases (c) e (d) restringiu-se às áreas de Ciências da Computação e Engenharia.  A \textit{string} de busca utilizada foi: \textsf{((narrativ* OR storytelling OR ``digital storytelling'') AND (``second screen'' OR multiscreen OR interactive OR transmedia OR crossmedia OR ``cross media''))}. A \autoref{tab-queries2} apresenta o protocolo utilizado em cada base de dados.

  \begin{table}[htb]
    \ABNTEXfontereduzida
    \caption[\textit{Strings} de busca]{\textit{Strings} de busca.}
    \label{tab-queries2}
    \begin{tabular}{p{2.0cm}|p{9.3cm}|p{2.8cm}}
      \textbf{Fonte} & \textbf{\textit{String} de busca} & \textbf{Nota} \\
      \hline
      ACM & Title:((narrativ* OR storytelling OR ``digital storytelling'') AND (``second screen'' OR multiscreen OR interactive OR transmedia OR crossmedia OR ``cross media'')) OR Abstract:((narrativ* OR storytelling OR ``digital storytelling'') AND (``second screen'' OR multiscreen OR interactive OR transmedia OR crossmedia OR ``cross media'')) & Busca em ``Advanced Search'', filtro de data adicionado manualmente \\
      \hline
      IEEE & (narrativ* OR storytelling OR ``digital storytelling'') AND (``second screen'' OR multiscreen OR interactive OR transmedia OR crossmedia OR ``cross media'') & Busca em ``Command Search'', filtro de data adicionado manualmente \\
      \hline
      ScienceDirect & pub-date > 2008 and tak((narrativ* OR storytelling OR ``digital storytelling'') AND (``second screen'' OR multiscreen OR interactive OR transmedia OR crossmedia OR ``cross media'')) [All Sources(Computer Science,Engineering)]) & Busca em ``Advanced search'', filtros ``pub-date'' e ``All Sources'' adicionados manualmente \\
      \hline
      Scopus & TITLE-ABS ( ( narrativ*  OR  storytelling  OR  ``digital storytelling'' )  AND  ( ``second screen''  OR  multiscreen  OR  interactive  OR  transmedia  OR  crossmedia  OR  ``cross media'' ) )  AND  ( LIMIT-TO ( SUBJAREA ,  ``COMP'' )  OR  LIMIT-TO ( SUBJAREA ,  ``ENGI'' ) )  AND  ( LIMIT-TO ( PUBYEAR ,  2016 )  OR  LIMIT-TO ( PUBYEAR ,  2015 )  OR  LIMIT-TO ( PUBYEAR ,  2014 )  OR  LIMIT-TO ( PUBYEAR ,  2013 )  OR  LIMIT-TO ( PUBYEAR ,  2012 )  OR  LIMIT-TO ( PUBYEAR ,  2011 )  OR  LIMIT-TO ( PUBYEAR ,  2010 )  OR  LIMIT-TO ( PUBYEAR ,  2009 ) ) & Busca em ``Advanced search'', filtros ``LIMIT-TO'' adicionados manualmente \\
    \end{tabular}
  \end{table}

  Com o propósito de identificar os critérios de inclusão, exclusão e extração, foram pensadas maneiras de se classificar os artigos a partir da estrutura ``O artigo trata de (categoria) para (objetivo) narrativas transmídia'', como pode ser visto na \autoref{diag-trata-de-para}.
  As categorias utilizadas como critérios de inclusão (CI) foram: (1) conceitos, apresentando ideias ou abstrações (2) processos, propondo fluxos ou sequências de ações; (3) ferramentas, mostrando softwares desenvolvidos (4) sincronismo, tratando de manter dados consistentes e (5) estudo de caso, apresentando provas de conceito ou implementações de terceiros. Essas categorias possibilitam atingir algum dos objetivos, que foram utilizados como critérios de extração: (a) planejar, (b) implementar, (c) analisar ou (d) adaptar narrativas transmídia.

  Também foram identificados dois critérios de exlusão (CE): (1) o artigo não está relacionado ao tema narrativas transmídia e (2) o resumo do artigo não esclarece o aspecto transmídia desenvolvido no trabalho.

  \begin{figure}[htb]
    \caption{\label{diag-trata-de-para}Estrutura de classificação}
    \begin{center}
      \smartdiagramset{font=\scriptsize, module minimum width=2.5cm, uniform arrow color=true, module x sep=4.0cm, back arrow disabled, uniform color list=white for 3 items}
      \smartdiagram[flow diagram:horizontal]{Conceitos \\ Processos \\ Ferramentas \\ Sincronismo de dados \\ Estudo de caso \\, Planejar \\ Implementar \\ Analisar \\ Adaptar, Narrativas transmídia} \\
    \end{center}
  \end{figure}

  A pesquisa inicial resultou em 1432 trabalhos, sendo 100 desses duplicados por conta da utilização de fontes de pesquisa que são agregadoras de publicações, como Scopus e ScienceDirect. A atividade de identificação e seleção dos estudos resumiu-se à leitura rápida dos títulos e resumos para associá-los aos critérios de inclusão ou exclusão. Como pode ser observado na \autoref{tab-article-selection}, dos 97 trabalhos selecionados, 21 (1,47\%) apresentam conceitos, 10 (0,70\%) propõem processos, somente 13 (0,90\%) mostram ferramentas, apenas um (0,07\%) trata sincronismo de dados e a grande maioria dos trabalhos (3,63\%) apresentam estudos de caso.

  \begin{table}[htb]
    \ABNTEXfontereduzida
    \caption[Etapa de seleção]{Etapa de seleção.}
    \label{tab-article-selection}
    \begin{tabular}{p{3.0cm}|p{2.0cm}|p{1.0cm}|p{1.0cm}|p{1.0cm}|p{1.0cm}|p{1.0cm}|p{2.0cm}}
      \textbf{Fonte} & \textbf{Resultantes da busca} & \textbf{CI1} & \textbf{CI2} & \textbf{CI3} & \textbf{CI4} & \textbf{CI5} & \textbf{Selecionados}  \\
      \hline
      ACM Digital Library & 240 & 9 & 4 & 6 & 1 & 12 & 32 \\
      \hline
      IEEE Xplore & 187 & 0 & 1 & 2 & 0 & 7 & 10 \\
      \hline
      ScienceDirect & 35 & 0 & 1 & 1 & 0 & 3 & 5 \\
      \hline
      Scopus & 970 & 12 & 4 & 4 & 0 & 30 & 50 \\
      \hline
      \textbf{Total} & \textbf{1432} & \textbf{21} & \textbf{10} & \textbf{13} & \textbf{1} & \textbf{52} & \textbf{97} \\
    \end{tabular}
  \end{table}

  % ----------------------------------------------------------
  % Resultados
  % ----------------------------------------------------------
  % Apresentar os resultados da extração
  \section{Resultados}
  \label{sec-resultados}

  Devido à quantidade de documentos encontrados não permitir sua exposição no corpo desse trabalho, foram produzidas tabelas para representar as classificações das categorias utilizadas na etapa de seleção e dos objetivos utilizados na extração dos \textit{papers} durante o processo de revisão. A tabela com todos os 97 artigos divididos por categorias pode ser encontrada no \autoref{sec-anexos}. Aqui, a \autoref{tab-trata-de} apresenta a classificação e a quantidade de documentos pertencentes a cada categoria, e a \autoref{tab-para} mostra os objetivos e quantos trabalhos pertencem a cada um deles.

  \begin{table}[htb]
    \ABNTEXfontereduzida
    \caption[Categorias de classificação]{Categorias de classificação.}
    \label{tab-trata-de}
    \begin{center}
      \begin{tabular}{p{3.0cm}|p{2.0cm}}
        \textbf{Trata de} & \textbf{Artigos} \\
        \hline
        Conceitos & 21 \\
        \hline
        Processos & 10 \\
        \hline
        Ferramentas & 13\\
        \hline
        Sincronismo & 1\\
        \hline
        Estudo de Caso & 52\\
        \hline
        \textbf{Total} & \textbf{97} \\
      \end{tabular}
    \end{center}
  \end{table}

  \begin{table}[htb]
    \ABNTEXfontereduzida
    \caption[Objetivos]{Objetivos.}
    \label{tab-para}
    \begin{center}
      \begin{tabular}{p{3.0cm}|p{2.0cm}}
        \textbf{Objetivo} & \textbf{Artigos} \\
        \hline
        Planejar & 20 \\
        \hline
        Implementar & 52 \\
        \hline
        Analisar & 14 \\
        \hline
        Adaptar & 11 \\
        \hline
        \textbf{Total} & \textbf{97} \\
      \end{tabular}
    \end{center}
  \end{table}

  % ----------------------------------------------------------
  % Discussão
  % ----------------------------------------------------------
  % Trabalhar a questão de que foram encontrados muitos resultados duplicados, muitos estudos de caso, somente uma pesquisa abordando a questão da sincronização dos dados
  % Comentar a quantidade de trabalhos que apresentam estudos de caso
  % Muitos jogos de realidade aumentada em diversas áreas do conhecimento
  \section{Discussão}
  \label{sec-discussao}

  Nessa seção são analisados os resultados encontrados na revisão sistemática, a fim de responder às questões de pesquisa elaboradas na \autoref{subsec-planejamento}.

  \subsection{Quais tópicos são mais pesquisados sobre narrativas transmídias?}

  Por meio da \autoref{tab-trata-de}, é possível perceber que o estudo de caso é o trabalho mais publicado, seguido da apresentação de conceitos. É possível que essas categorias sejam mais desenvolvidas pelos pesquisadores por conta da falta de trabalhos de referência que especifiquem aplicações de narrativas transmídia na Computação, o que pode levar à documentação de experiências práticas a fim de testar ideias (estudo de caso) que não necessariamente são oriundas da Computação ou conceituar termos e abstrações sobre narrativas transmídia na Computação.

  Em \citeonline{dionisio_2015}, por exemplo, pode-se ver como os autores apresentam um estudo de caso em que se analisa como narrativas transmídia auxiliam a inserir novos jogadores no contexto dos jogos de tabuleiro. Já \citeonline{velikovsky_2014}, introduz conceitos para analisar narrativas transmídia e reforçar o conceito de videogame como arte.

  A pouca ênfase em estudos que trabalham o sincronismo de dados nas narrativas transmídia é um dado que pode causar certo estranhamento. Considerando a narrativa transmídia parte de um sistema distribuído, pois ela é espalhada através de vários meios, o sincronismo de dados é um tema tratado e com processos e ferramentas bem desenvolvidos, como aponta \citeonline{sementille_1999}.

  \subsection{Quais os objetivos mais apresentados?}

  Como pode ser observado na \autoref{tab-trata-de}, é notória a quantidade de artigos que têm como objetivo principal implementar uma narrativa transmídia. A título de exemplo, \citeonline{tomi_2013} relatam como se deu a implementação de um livro de realidade aumentada para auxiliar crianças em fase de pré-escola a aprender os números. Planejamento, também tratado como \textit{design} ou criação, de narrativas transmídia figura entre os objetivos mais publicados. \citeonline{bonsignore_2012} discute o \textit{design} e jogabilidade dos jogos de realidade aumentada como espaços participativos de design.

  As narrativas transmídia possuem diversas área de aplicação, e as publicações em Computação notoriamente apresentam trabalhos nas áreas de artes \cite{santorineos_2009,ha_2012,jung_2012,katifori_2014}, educação \cite{raybourn_2014,xiao_2013,mcauliffe_2011,ballagas_2011} e entretenimento \cite{willis_2013,choi_2010,evans_2014,nandakumar_2014,holler_2014,murakami_2015}.

  % ----------------------------------------------------------
  % Conclusão
  % ----------------------------------------------------------
  % A revisão sistemática mostra como a pesquisa sobre narrativas transmídia é multidisciplinar
  % As pesquisas sobre narrativas transmídia em Computação continuam em alta, especialmente nas aplicações em jogos e segunda tela
  \section{Conclusão}

  A convergência produzida a partir da capacidade de agregação da Internet promoveu o surgimento das narrativas transmídia, que dispersam o enredo em diversas plataformas. Como uma maneira de contar histórias apoiada fortemente nas tecnologias digitais, as pesquisas em narrativas transmídia na Computação são amplas e variadas e uma classificação e organização dessa produção pode ser um mecanismo útil para compreender seu atual estado da arte.

  Este estudo apresenta uma revisão sistemática sobre narrativas transmídia em Computação a partir de objetivos e questões de pesquisas elaborados na \autoref{sec-metodos}. Os resultados da pesquisa, apresentados na \autoref{sec-resultados} e discutidos na \autoref{sec-discussao}, mostram que as pesquisas se caracterizam por apresentarem grandes quantidades de estudos de caso e conceitos com a finalidade de implementar e planejar narrativas transmídia. A utilização de outras fontes de pesquisa produziriam diferentes resultados, bem como o uso de categorias de classificação e de objetivos não abordados nesse trabalho, o que permite o desenvolvimento de futuras pesquisas sobre o tema.

  Esse trabalho visa contribuir para a discussão de uma possível classificação e categorização das pesquisas em narrativas transmídia para auxiliar a comunidade acadêmica a compreender e identificar temas relevantes de pesquisa.

  % ---
  % Finaliza a parte no bookmark do PDF, para que se inicie o bookmark na raiz
  % ---
  \bookmarksetup{startatroot}%
  % ---

  % ----------------------------------------------------------
  % ELEMENTOS PÓS-TEXTUAIS
  % ----------------------------------------------------------
  \postextual

  % ---
  % Título e resumo em língua estrangeira
  % ---

  % titulo em inglês
  \titulo{Research topics on transmedia storytelling}
  \emptythanks
  \maketitle

  % resumo em português
  \renewcommand{\resumoname}{Abstract}
  \begin{resumoumacoluna}
    \begin{otherlanguage*}{english}
      This paper discuss a systematic review on research topics in transmedia storytelling. The chosen sources were: ACM, IEEE, ScienceDirect, Scopus

      \vspace{\onelineskip}

      \noindent
      \textbf{Keywords}: latex. abntex.
    \end{otherlanguage*}
  \end{resumoumacoluna}

  % ---

  % ----------------------------------------------------------
  % Referências bibliográficas
  % ----------------------------------------------------------
  \bibliography{mono}
  % ----------------------------------------------------------
  % Anexos
  % ----------------------------------------------------------
  \cftinserthook{toc}{AAA}
  % ---
  % Inicia os anexos
  % ---
  %\anexos
  \begin{anexosenv}

    % ---
    \chapter{Artigos resultantes da revisão sistemática}
    % ---
    \tiny
    \centering
    \label{sec-anexos}
    \begin{longtable}{p{6.0cm}|p{3.0cm}|p{.6cm}|p{2.0cm}|p{2.0cm}}
      \textbf{Título} & \textbf{Autor} & \textbf{Ano} & \textbf{Trata de} & \textbf{Para} \\
      \hline
      An Interactive Mobile Augmented Reality Magical Playbook: Learning Number with the Thirsty Crow&Azfar Bin Tomi and Dayang Rohaya Awang Rambli&2013&Estudo de Caso & Planejar \\
      \hline
      From tradition to emerging practice: A hybrid computational production model for Interactive Documentary &Insook Choi&2010&Processo & Implementar \\
      \hline
      An Alternate Reality Game for Facility Resilience (ARGFR) &Jing Pan and Xing Su and Zheng Zhou&2015&Estudo de Caso & Implementar \\
      \hline
      Technology-enhanced role-play for social and emotional learning context \& Intercultural empathy &Mei Yii Lim and Karin Leichtenstern and Michael Kriegel and Sibylle Enz and Ruth Aylett and Natalie Vannini and Lynne Hall and Paola Rizzo&2011&Ferramenta & Implementar\\
      \hline
      Them and Us: An indoor pervasive gaming experience &Alan Chamberlain and Fernando Martínez-Reyes and Rachel Jacobs and Matt Watkins and Robin Shackford&2013&Estudo de Caso & Implementar\\
      \hline
      Kids in Fairytales: Experiential and Interactive Storytelling in Children's Libraries&Kang, Seokbin and Lee, Youngwoon and Lee, Suwoong&2015&Estudo de Caso & Analisar\\
      \hline
      I-Eng: An Interactive Toy for Second Language Learning&Jeong, Hayeon and Saakes, Daniel Pieter and Lee, Uichin&2015&Estudo de Caso & Implementar\\
      \hline
      Incorporating Fictionality into the Real Space: A Case of Enhanced TCG&Takahashi, Monami and Irie, Keisuke and Sakamoto, Mizuki and Nakajima, Tatsuo&2015&Estudo de Caso & Implementar\\
      \hline
      HideOut: Mobile Projector Interaction with Tangible Objects and Surfaces&Willis, Karl D. D. and Shiratori, Takaaki and Mahler, Moshe&2013&Ferramenta & Implementar\\
      \hline
      Flow Theory, Evolution \& Creativity: Or, ``Fun \& Games''&Velikovsky, JT&2014&Conceito & Analisar\\
      \hline
      Fall of Humans: Interactive Tabletop Games and Transmedia Storytelling&Dionisio, Mara and Gujaran, Aditya and Pinto, Miguel and Esteves, Augusto&2015&Estudo de Caso & Implementar\\
      \hline
      Expanding the Comics Canvas: GPS Comics&Samanci, Ozge and Tewari, Anuj&2012&Estudo de Caso & Adaptar\\
      \hline
      Evaluating Enjoyment Within Alternate Reality Games&Macvean, Andrew P. and Riedl, Mark O.&2011&Processo & Analisar\\
      \hline
      Engaging Theatre Audiences Before the Play: The Design of Playful Interactive Storytelling Experiences&Dima, Mariza&2013&Conceito & Adaptar\\
      \hline
      Don'T Open That Door: Designing Gestural Interactions for Interactive Narratives&Clifton, Paul and Caldwell, Jared and Kulka, Isaac and Fassone, Riccardo and Cutrell, Jonathan and Terraciano, Kevin and Murray, Janet and Mazalek, Ali&2013&Processo & Implementar\\
      \hline
      Designing Tangible Interfaces to Support Expression and Sensemaking in Interactive Narratives&Chu, Jean Ho&2015&Conceito & Planejar\\
      \hline
      Designing iDTV Applications Through Interactive Storyboards&Ara\'{u}jo, Eduardo Cruz and Soares, Luiz Fernando Gomes&2014&Ferramenta & Planejar\\
      \hline
      Designing Alternate Reality Games&Bonsignore, Elizabeth&2012&Conceito & Planejar\\
      \hline
      Design Tactics for Authentic Interactive Fiction: Insights from Alternate Reality Game Designers&Bonsignore, Elizabeth and Moulder, Vicki and Neustaedter, Carman and Hansen, Derek and Kraus, Kari and Druin, Allison&2014&Conceito & Planejar\\
      \hline
      Companion Apps for Long Arc TV Series: Supporting New Viewers in Complex Storyworlds with Tightly Synchronized Context-sensitive Annotations&Nandakumar, Abhishek and Murray, Janet&2014&Sincronismo & Adaptar\\
      \hline
      An Interactive Multimedia Framework for Digital Heritage Narratives&Adabala, Neeharika and Datha, Naren and Joy, Joseph and Kulkarni, Chinmay and Manchepalli, Ajay and Sankar, Aditya and Walton, Rebecca&2010&Ferramenta & Implementar\\
      \hline
      Augmented Creativity: Bridging the Real and Virtual Worlds to Enhance Creative Play&Z\"{u}nd, Fabio and Ryffel, Mattia and Magnenat, St{\'e}phane and Marra, Alessia and Nitti, Maurizio and Kapadia, Mubbasir and Noris, Gioacchino and Mitchell, Kenny and Gross, Markus and Sumner, Robert W.&2015&Processo & Implementar\\
      \hline
      Canyons, Deltas and Plains: Towards a Unified Sculptural Model of Location-based Hypertext&Millard, David E. and Hargood, Charlie and Jewell, Michael O. and Weal, Mark J.&2013&Conceito & Planejar\\
      \hline
      Adapting Historical Drama for the Web: A Model for Metadata Backed Publishing of Historical Drama Programmes&Davies, Rosamund and Rissen, Paul and Jewell, Michael O.&2013&Processo & Adaptar\\
      \hline
      WeQuest: Scalable Alternate Reality Games Through End-user Content Authoring&Macvean, Andrew and Hajarnis, Sanjeet and Headrick, Brandon and Ferguson, Aziel and Barve, Chinmay and Karnik, Devika and Riedl, Mark O.&2011&Ferramenta & Implementar\\
      \hline
      Transmedia in the Classroom: Breaking the Fourth Wall&Teske, Paul R. J. and Horstman, Theresa&2012&Estudo de Caso & Adaptar\\
      \hline
      Transcending Transmedia: Emerging Story Telling Structures for the Emerging Convergence Platforms&Murray, Janet H.&2012&Conceito & Planejar\\
      \hline
      The Remediation of Nosferatu: Exploring Transmedia Experiences&Ghellal, Sabiha and Morrison, Ann and Hassenzahl, Marc and Schaufler, Benjamin&2014&Estudo de Caso & Implementar\\
      \hline
      The Reading Glove: Designing Interactions for Object-based Tangible Storytelling&Tanenbaum, Joshua and Tanenbaum, Karen and Antle, Alissa&2010&Estudo de Caso & Implementar\\
      \hline
      The ABC's of ARGs: Alternate Reality Games for Learning&Olbrish, Koreen&2011&Conceito & Planejar\\
      \hline
      Storied Numbers: Supporting Media-rich Data Storytelling for Television&Robinson, Susan J. and Williams, Graceline and Parnami, Aman and Kim, Jinhyun and McGregor, Emmett and Chandler, Dana and Mazalek, Ali&2014&Ferramenta & Implementar\\
      \hline
      Social Interaction for Interactive Storytelling&de Lima, Edirlei Soares and Feij\'{o}, Bruno and Pozzer, Cesar T. and Ciarlini, Angelo E. M. and Barbosa, Simone D. J. and Furtado, Antonio L. and da Silva, Fabio A. Guilherme&2012&Ferramenta & Implementar\\
      \hline
      Silent Mutations: Physical-digital Interactions in Spaces&R{\'e}bola Winegarden, Claudia and Komor, Nicholas and Gilliland, Scott M.&2010&Conceito & Analisar\\
      \hline
      PuzzleTale: A Tangible Puzzle Game for Interactive Storytelling&Shen, Yang Ting and Mazalek, Ali&2010&Estudo de Caso & Implementar\\
      \hline
      Multisensory, pervasive, immersive: Towards a new generation of documents&Robinson, L.&2015&Conceito & Planejar\\
      \hline
      Designing interactive narratives for mobile augmented reality&Nam, Y.&2015&Processo & Implementar \\
      \hline
      Thresholds of transmedia storytelling: Applying G{\'e}rard Genette's paratextual theory to the 39 clues series for young readers&Nottingham-Martin, A.&2014&Conceito & Analisar \\
      \hline
      Beyond intuitive UI: Design considerations for attention, rhythm, and weight&Tan, A.&2014&Conceito & Analisar\\
      \hline
      Educational uses of transmedia storytelling. The ancestral letter&Rodriguez Illera, J.L. and Molas Castells, N.&2014&Estudo de Caso&Implementar \\
      \hline
      Immersive interactive narratives in augmented reality games&Viana, B.S. and Nakamura, R.&2014&Conceito & Planejar\\
      \hline
      Digitally augmented narratives for physical artifacts&Bellucci, A. and Diaz, P. and Aedo, I.&2014&Processo & Planejar\\
      \hline
      Transmedia storytelling and the creation of a converging space of educational practices&Rodr{\'i}gues, P.a  b  and Bidarra, J.a  c &2014&Conceito & Implementar\\
      \hline
      Narratives of augmented worlds&Shilkrot, R.a  and Montfort, N.b  and Maes, P.a &2014&Conceito & Planejar\\
      \hline
      StoryCube: Supporting children's storytelling with a tangible tool&Wang, D. and He, L. and Dou, K.&2014&Estudo de Caso & Implementar\\
      \hline
      The Bridge - A transmedia dialogue between TV, Film and Gaming&Elias, H.&2014&Conceito & Analisar\\
      \hline
      CHESS: Personalized Storytelling Experiences in Museums&Katifori, A.a  and Karvounis, M.a  and Kourtis, V.a  and Kyriakidi, M.a  and Roussou, M.a  and Tsangaris, M.a  and Vayanou, M.a  and Ioannidis, Y.a  and Balet, O.b  and Prados, T.b  and Keil, J.c  and Engelke, T.c  and Pujol, L.d &2014&Ferramenta & Adaptar\\
      \hline
      NARRATIVES: Geolocative cinema application&Nagler, S.a  and Hackett, M.b  and Hicks, A.a  and Zachkarko, K.b &2014&Estudo de Caso & Adaptar\\
      \hline
      The GamiMedia model: Gamifying content culture&Sakamoto, M. and Nakajima, T.&2014&Conceito & Analisar\\
      \hline
      The Malthusian Paradox: performance in an alternate reality game&Evans, E. and Flintham, M. and Martindale, S.&2014&Estudo de Caso & Implementar\\
      \hline
      A new paradigm for serious games: Transmedia learning for more effective training and education&Raybourn, E.M.&2014&Conceito & Planejar\\
      \hline
      The PC3 framework: A formal lens for analyzing interactive narratives across media forms&Magerko, B.&2014&Processo & Analisar\\
      \hline
      Revisiting history: Using alternate reality games to tell a centuryold tale&Lynch, R. and Mallon, B. and Connolly, C.&2014&Estudo de Caso & Implementar\\
      \hline
      Time tremors: Developing transmedia gaming for children&Holler, C.a  and Tindale, A.b  and Crowe, P.a  and Diamond, S.b  and Mayhew, A.a &2014&Estudo de Caso & Implementar\\
      \hline
      Walking into the Past: Design mobile app for the geo-referred and the multimodal user experience in the context of cultural heritage&Bollini, L.a  and De Palma, R.b  and Nota, R.a &2013&Estudo de Caso & Implementar\\
      \hline
      Augmenting Yu-Gi-Oh! Trading card game as persuasive transmedia storytelling&Sakamoto, M. and Nakajima, T.&2013&Conceito & Planejar\\
      \hline
      Digital storytelling within virtual environments: ``The Battle of Thermopylae''&Christopoulos, D. and Mavridis, P. and Andreadis, A. and Karigiannis, J.N.&2013&Estudo de Caso & Implementar\\
      \hline
      Language innovations in documentary 33 &Ren{\'o}, D.P. and Ruiz, S.L.&2013&Estudo de Caso & Analisar\\
      \hline
      Designing enhanced daily digital artifacts based on the analysis of product promotions using fictional animation stories&Sakamoto, M.a  and Nakajima, T.a  and Akioka, S.b &2013&Conceito & Planejar\\
      \hline
      Storytelling: An ancient human technology and critical-creative pedagogy for transformative learning&Kalogeras, S.&2013&Conceito & Planejar\\
      \hline
      Slideware2.0: A prototype of presentation system by integrating web2.0 and second screen to promote education communication&Xiao, R.a  and Wu, Z.b  and Sugiura, K.a &2013&Ferramenta & Implementar\\
      \hline
      Multiple-channel video installation as a precursor to transmedia-based artAb&Wu, G.a  and Gough, P.b  and De Berigny Wall OnacloV, C.c &2012&Estudo de Caso & Analisar\\
      \hline
      Tagtool: A collaborative virtual workspace for visual expression&Pintaric, T.a  and Dorninger, M.b  and Csisinko, M.a  and Dorninger, J.b  and Pilz, F.a  and Fritz, M.b  and Norden, M.a &2012&Estudo de Caso & Implementar\\
      \hline
      How to preserve inspirational environments that once surrounded a poet? Immersive 360 video and the cultural memory of Charles Causley's poetry&Kwiatek, K.&2012&Estudo de Caso & Implementar\\
      \hline
      Alternate reality game for university-level computer science education&Hakulinen, L.&2012&Estudo de Caso & Planejar\\
      \hline
      Experiments with the internet of things in museum space: QRator&Hudson-Smith, A. and Gray, S. and Ross, C. and Barthel, R. and De Jode, M. and Warwick, C. and Terras, M.&2012&Estudo de Caso & Adaptar\\
      \hline
      i-Theatre: Tangible interactive storytelling&Mu{\~n}oz, J. and Marchesoni, M. and Costa, C.&2012&Estudo de Caso & Implementar\\
      \hline
      Cyclic.: An interactive performance combining dance, graphics, music and kinect-technology&Jung, D.a  and Jensen, M.H.b  and Laing, S.c  and Mayall, J.d &2012&Estudo de Caso & Implementar\\
      \hline
      Making Learning Active with Interactive Whiteboards, Podcasts, and Digital Storytelling in ELL Classrooms&Hur, J.W. and Suh, S.&2012&Estudo de Caso & Adaptar\\
      \hline
      Miniature alive: Augmented reality-based interactive DigiLog experience in miniature exhibition&Ha, T.a  and Kim, K.b  and Park, N.a  and Seo, S.b  and Woo, W.a &2012&Estudo de Caso & Implementar\\
      \hline
      Mobile Urban Drama: Interactive storytelling in real world environments&Hansen, F.A. and Kortbek, K.J. and  GrønbÆk, K. &2012&Ferramenta & Implementar\\
      \hline
      Using an alternate reality game to increase physical activity and decrease obesity risk of college students&Johnston, J.D.a  and Massey, A.P.b  and Marker-Hoffman, R.L.a &2012&Estudo de Caso & Implementar\\
      \hline
      Scaling mobile alternate reality games with geo-location translation&Hajarnis, S. and Headrick, B. and Ferguson, A. and Riedl, M.O.&2011&Processo & Implementar\\
      \hline
      HIP-storytelling: Hand interactive projection for storytelling&Melo, N.a  and Salgado, P.a  and Iurgel, I.a  b  c  and Branco, P.a  c &2011&Estudo de Caso & Implementar\\
      \hline
      Mysteries and heroes: Using Imaginative Education to engage middle school learners in engineering&McAuliffe, L.a  and Ellis, G.W.a  and Ellis, S.K.a  and Huff, I.a  and McGinnis-Cavanaugh, B.b &2011&Estudo de Caso & Implementar\\
      \hline
      Electric agents: Combining television and mobile phones for an educational game&Ballagas, R.a  and Revelle, G.b  and Buza, K.a  and Horii, H.a  and Mori, K.a  and Raffle, H.a  and Spasojevic, M.a  and Go, J.a  and Cook, K.c  and Reardon, E.c  and Tsai, Y.-T.a  and Paretti, C.a &2011&Estudo de Caso & Implementar\\
      \hline
      Touching and being touched: Embodied experiences with belongings carried through memory and time&Veronesi, F.a  and Bongers, B.b &2010&Estudo de Caso & Implementar\\
      \hline
      Playing With Poetry a Portuguese Transmedia experience and a serious ARG&Gouveia, P.&2010&Estudo de Caso & Planejar\\
      \hline
      Harnessing ``e'' in storyworlds: Engage, enhance, experience, entertain&Norrington, A.&2010&Estudo de Caso & Implementar\\
      \hline
      Bridging media with the help of players&Nitsche, M. and Drake, M. and Murray, J.&2009&Estudo de Caso & Implementar\\
      \hline
      Introducing multiple interaction devices to interactive storytelling: Experiences from practice&Kurdyukova, E. and André, E. and Leichtenstern, K.&2009&Estudo de Caso & Implementar\\
      \hline
      Immersive mixed media augmented reality applications and technology&Kuchelmeister, V. and Shaw, J. and McGinity, M. and Del Favero, D. and Hardjono, A.&2009&Ferramenta & Implementar\\
      \hline
      Noon - A secret told by objects&Martins, T.a  b  and Sommerer, C.a  and Mignonneau, L.a  and Correia, N.b &2009&Estudo de Caso & Implementar\\
      \hline
      Action room: A low-cost hypermedia platform for experimental performances and spectacles&Santorineos, M. and Zoi, S.&2009&Estudo de Caso & Implementar\\
      \hline
      Arguing for multilingual motivation in web 2.0: An evaluation of a large-scale European pilot&Hainey, T.a  and Connolly, T.a  and Stansfield, M.a  and Boyle, L.a  and Josephson, J.b  and O'Donovan, A.c  and Ortiz, C.R.d  and Tsvetkova, N.e  and Stoimenova, B.e  and Tsvetanova, S.f &2009&Estudo de Caso & Planejar\\
      \hline
      Playing for Real: Designing Alternate Reality Games for Teenagers in Learning Contexts&Bonsignore, Elizabeth and Hansen, Derek and Kraus, Kari and Visconti, Amanda and Ahn, June and Druin, Allison&2013&Estudo de Caso & Implementar\\
      \hline
      Left to Their Own Devices: Ad Hoc Genres and the Design of Transmedia Narratives&Hashimov, Elmar and McNely, Brian&2012&Estudo de Caso & Adaptar\\
      \hline
      Narrative Agency and User Experience in Transmedia Narratives: Brazilian Telenovelas Case&Murakami, Mariane Harumi&2015&Estudo de Caso & Adaptar\\
      \hline
      Mixing reality and magic at Disney theme parks&Mine, M.&2009&Ferramenta & Implementar\\
      \hline
      Alice's adventures in an immersive mixed reality environment&Nakevska, M. and Jun Hu and Langereis, G. and Rauterberg, M.&2012&Estudo de Caso & Implementar\\
      \hline
      Transmedia Storytelling and Online Representations -- Issues of Trust on the Internet&Zingerle, A. and Kronman, L.&2011&Estudo de Caso & Analisar\\
      \hline
      Narrative paradox and the design of alternate reality games (ARGs) and blogs&Gouveia, P.&2009&Estudo de Caso & Analisar\\
      \hline
      Avatoys: Hibrid system with real and digital puppets&Antonijoan, M. and Soler, D. and Miralles, D.&2014&Estudo de Caso & Implementar\\
      \hline
      Tweeting with the telly on!&Lochrie, M. and Coulton, P.&2012&Estudo de Caso & Analisar\\
      \hline
      Transitions: A Crossmedia Interaction Relevant Aspect&Delgado Preti, J.P. and Miyamaru, F. and Vilela Leite Filgueiras, L.&2014&Ferramenta & Implementar\\
      \hline
      The Ghost Club Storyscape: Designing for transmedia storytelling&Blumenthal, H. and Yan Xu and Mokashi, S. and Ramanujam, N. and Nunes, J. and Shemaka, R.G.&2011&Estudo de Caso & Implementar\\
      \hline
      iFiction: Mobile technology, new media, Mixed Reality and literary creativity in English teaching&Chinthammit, Winyu and Thomas, Angela&2012&Estudo de Caso & Implementar\\
      \hline
      Wish shop: Designing for a transmedia production model&Heejun Lim and Chungkon Shi&2012&Processo & Planejar\\
    \end{longtable}

  \end{anexosenv}

\end{document}
